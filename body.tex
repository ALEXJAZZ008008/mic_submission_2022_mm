\section{Introduction} \label{sec:introduction}
    \IEEEPARstart{M}{otion} modelling is a \gls{MC} technique, where time- or gate-dependence of \glss{DVF} are parameterised in terms of a \gls{SS}~\cite{McClelland2013}. \glss{MM} attempt to improve upon solely registering data, being more robust to noise, but also allow for the correction of unseen data. It has shown good promise in \gls{CT}~\cite{Li2007EnhancedModel}, \gls{MR}~\cite{Manke2002RespiratoryModels}
    and  combined \gls{PET}/\gls{MR}~\cite{Manber2016JointCorrection.}, but has not yet seen widespread adoption in clinical \gls{PET}, where \gls{PET}/\gls{CT} is far more common.
    
    \gls{RM} reduces resolution and degrades the accuracy of quantification in \gls{PET} by introducing blurring to the \gls{PET} volume and also misalignment between the \gls{PET} and \gls{CT}~\cite{Nehmeh2008a}. Most existing \gls{MC} methods rely on pair-wise registration of gated \gls{PET} volumes, this is a challenging problem due to the low contrast and high noise~\cite{Oliveira2014}. Respiratory \gls{MC} is an ideal problem area for the application of \glss{MM} as \glss{SS} are already commonly available from respiratory gating, such as acquired by \gls{RPM} or \gls{PCA}~\cite{Thielemans2011}.
    
    In previous work, the possibility of using \glss{MM} to \gls{MC} \gls{NAC} \gls{TOF} \gls{PET}, and warp a \gls{Mu-Map} from a position close to the mean respiratory position to each gate, was investigated. It was found that the combination of both the \gls{MM} and \gls{TOF} was sufficient to perform an \gls{AC} reconstruction with \gls{MC}, without introducing artefacts, while increasing resolution and quantification accuracy~\cite{Whitehead2019ImpactPET},~\cite{Whitehead2020PET/CTFields}. This work seeks to extend the method further through the use of a more modular framework, which allows for the fair comparison of different registration methods, both with and without \glss{MM}. Furthermore, this work uses more realistic simulation and count levels, compared to previous work (where more simple registration methods would fail). Additionally, this work strives to improve the \gls{Mu-Map} warping aspect of the previous method, by fixing the \gls{Mu-Map} at end inhalation (as opposed to the mean respiratory position). This is more clinically relevant but also challenging.
    
    A method incorporating \glss{MM} for dynamic \gls{PET}/\gls{CT}, was proposed and tested on clinical data in~\cite{Chan2018Non-RigidPET}. The work presented here, differentiates itself by firstly using a \gls{2D} \gls{SS}, rather than a \gls{1D} \gls{SS}, thus both inter- and intra-gate motion can be included in the model, at the expense that each gate contains fewer counts. Additionally, the group-wise method, presented here, makes use of an iterative \gls{MC} algorithm rather than using only a pair-wise method.

% \vspace{-0.3cm}

\section{Methods} \label{sec:methods}
    \subsection{XCAT Volume Generation} \label{sec:xcat_volume_generation}
        \gls{XCAT}~\cite{Segars2010} was used to generate $480$ volumes over a \SI{240}{\second} period using a respiratory trace, derived from \gls{MR} navigator patient data. The max displacement of \acrlong{AP} and \acrlong{SI} motion, was set to \SI{1.2}{\centi\metre} and \SI{2.0}{\centi\metre} respectively. Activity concentrations were derived from a static \gls{18F-FDG} patient scan. The \gls{FOV} included the base of the lungs, diaphragm and the top of the liver with a \SI{20}{\milli\metre} diameter spherical lesion (smaller than the max displacement, due to \gls{RM}) was placed into the base of the right lung (within the max displacement, due to \gls{RM}, of the diaphragm).
    
    % \vspace{-0.3cm}
    
    \subsection{PET Acquisition Simulation and Non-Attenuation Corrected Image Reconstruction} \label{sec:pet_acquisition_simulation_and_non_attenuation_corrected_image_reconstruction}
        \gls{PET} acquisitions were simulated (and reconstructed) using \gls{STIR}~\cite{Thielemans2012, Nikos2019} through \gls{SIRF}~\cite{Ovtchinnikov2017}, to forward project data using the geometry of a \gls{GE} Discovery $710$, but using a \gls{TOF} resolution of \SI{375}{\pico\second}. This \gls{TOF} resolution is higher than that of the $710$, but is closer to the newer \gls{GE} Signa \gls{PET}/\gls{MR} system. \gls{TOF} mashing was used to reduce computation time resulting in $13$ \gls{TOF} bins of size \SI{376.5}{\pico\second}. Attenuation was included using the relevant \glss{Mu-Map} generated by \gls{XCAT}. Pseudo-randoms and scatter were added. Randoms were added by summing the scaled mean value to each voxel of each volume prior to forward projection. Pseudo scatter was added by summing the scaled and smoothed mean \gls{Mu-Map} prior to forward projection, the smoothing parameter was optimised to give scatter which tapered at the same rate as in clinical data. A full scatter simulation was not performed due to software limitations.
        
        Noise was simulated, such that data matched an acquisition over \SI{120}{\second}, emulating a standard single bed position acquisition. The count rate was selected to match that of research scans, i.e. below that of diagnostic clinical scans. This count rate was selected as a 'worst case scenario'.
        
        A respiratory \gls{SS} was generated using \gls{PCA}~\cite{Thielemans2011}. The magnitude of this signal and its gradient, was used to gate data into $30$ respiratory bins using displacement gating ($10$ amplitude and $3$ gradient bins). Gates with fewer than $0.42$\% of the counts were discarded. For the purpose of the \gls{MM} fitting, \gls{SS} values were determined for the post-gated data by taking an average of the \gls{SS} values of data in each bin.
        
        Data were reconstructed, without \gls{AC}, using \gls{OSEM} with two full iterations and $24$ subsets~\cite{Hudson1994}.
    
    % \vspace{-0.3cm}
    
    \subsection{Registration} \label{sec:registration}
        Before being registered, each volume underwent pre-processing. Including replication of end-slices, transformation to be approximately normally distributed~\cite{Johnson2013} and post-smoothing. This pre-processing was only applied to intermediate data and was not used for the final output of the method.% Initially, because a breath hold \gls{Mu-Map} is the final target position for the \gls{MC} $10$ repeating slices are added to the top and bottom of each volume to allow space for the volumes to be registered into. First, the mean value was subtracted from each volume and then each voxel in the volume was divided by the standard deviation of the volume. Next a Yeo-Johnson transformation~\cite{Johnson2013} was applied to transform data to be more Gaussian like, this acted as a pseudo histogram normalisation. Finally data underwent Gaussian smoothing.
        
        Two registration methods were examined in this work. Firstly, pair-wise registration, where the reference position was selected as the gate with the highest number of counts. All other gates were registered to it. Secondly, group-wise registration, where after an initial pair-wise registration step, the \glss{DVF} generated had the inverse mean of all \glss{DVF} composed with them, before a new reference volume was resampled. Registration to the new reference volume, followed by the inverse mean composition and resample, continued for a set number of iterations. NiftyReg~\cite{Modat2010} was used to perform registrations using a B-spline parameterisation. The Gaussian smoothing \gls{FWHM}, \acrlong{CPG} spacing of the B-spline coefficients, \acrlong{BE} regularisation term weight and number of iterations were tuned using a grid search.
    
    % \vspace{-0.3cm}
    
    \subsection{Motion Model Estimation} \label{sec:motion_model_estimation}
        If a \gls{MM} was used, then it was fit as a direct \acrlong{RCM} on the \glss{DVF} from~\Fref{sec:registration} and the \gls{SS} from~\Fref{sec:pet_acquisition_simulation_and_non_attenuation_corrected_image_reconstruction}. A weighted \acrlong{LR} was used, where the weighting was taken based on the number of counts in each gate. Once a \gls{MM} was fit, new \glss{DVF} were generated for each gate, using the \gls{SS} values used to fit the \gls{MM}. For group-wise registration, \gls{MM} fitting occurred between iterations, the \glss{DVF} generated by the \gls{MM} were used to resample the new target volume at each iteration.
    
    % \vspace{-0.3cm}
    
    \subsection{Attenuation Map Warping} \label{sec:attenuation_map_warping}
        A \gls{Mu-Map} at end inhalation was selected from the \glss{Mu-Map} generated by \gls{XCAT}. The \gls{PET} volume from the previous step was then registered to this \gls{Mu-Map}, and the resulting \glss{DVF} were composed with the \glss{DVF} from the last iteration of the \gls{MC} method, and a new volume resampled. The inverse of these \glss{DVF}, were then used to warp the \gls{Mu-Map} to each gate.
    
    % \vspace{-0.3cm}
    
    \subsection{Motion Corrected Image Reconstruction with AC} \label{sec:attenuation_corrected_image_reconstruction}
        Data were re-reconstructed with \gls{AC}, using the \glss{Mu-Map} from~\Fref{sec:attenuation_map_warping}. The same reconstruction parameters as in~\Fref{sec:attenuation_corrected_image_reconstruction} were used. \gls{MC} was then applied to data following~\Fref{sec:registration},~\Fref{sec:motion_model_estimation} and~\Fref{sec:attenuation_map_warping}. Volumes were post-filtered using a Gaussian smoothing, with a \gls{FWHM} of \SI{6.39}{\milli\metre} in the transverse plane (equivalent to three voxels) and \SI{3.27}{\milli\metre} (equivalent to one voxel) in the axial direction.
    
    % \vspace{-0.3cm}
    
    \subsection{Evaluation} \label{sec:evaluation}
        In addition to the reconstructions performed in~\Fref{sec:attenuation_corrected_image_reconstruction}, data were also reconstructed without \gls{MC}, using either a sum of all \glss{Mu-Map} (to emulate an \gls{AV-CCT}), or the end inhalation \gls{Mu-Map}. For the present evaluation, the volumes without \gls{MC} were registered to the position of the end inhalation \gls{Mu-Map}. Additionally, \glss{DVF} generated by each method were also applied to noiseless data for visual analysis.
        
        Comparisons used included: A profile over the lesion, \gls{SUV}\textsubscript{max} and \gls{SUV}\textsubscript{peak} (defined following \gls{EANM} guidelines~\cite{Boellaard2015FDG2.0}).

% \vspace{-0.3cm}

\section{Results} \label{sec:results}
    % \begin{figure}
        % \vspace{-0.0cm}
        
    %     \centering
        
    %     \includegraphics[width=1.0\linewidth]{figures/visual_analysis.png}
        
        % \vspace{-0.3cm}
        
    %     \captionsetup{singlelinecheck=false, justification=centering}
    %     \caption{First column contains \gls{AC} \gls{MC} reconstructions and the second column contains the result of applying the final \gls{MC} on the original XCAT images (for easier assessment of the accuracy of the estimated \glss{DVF}); ungated static \gls{CT}, ungated \gls{AV-CCT}, pair-wise, pair-wise \gls{MM}, group-wise, group-wise \gls{MM}. Colour map ranges are consistent for all images in each column.}
        
    %     \label{fig:visual_analysis}
        
        % \vspace{-0.3cm}
    % \end{figure}
    
    

% \vspace{-0.3cm}

\section{Discussion and Conclusions} \label{sec:discussion_and_conclusions}
    
    